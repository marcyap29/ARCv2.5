\documentclass[11pt,letterpaper]{article}

% Page layout
\usepackage[margin=1in]{geometry}
\usepackage{fancyhdr}
\usepackage{lastpage}

% Typography and fonts
\usepackage[utf8]{inputenc}
\usepackage[T1]{fontenc}
\usepackage{lmodern}
\usepackage{microtype}
\usepackage{setspace}

% Graphics and colors
\usepackage{xcolor}
\usepackage{graphicx}

% Tables
\usepackage{booktabs}
\usepackage{tabularx}
\usepackage{array}
\usepackage{multirow}
\usepackage{longtable}

% Math
\usepackage{amsmath}
\usepackage{amssymb}

% Links and references
\usepackage{hyperref}
\hypersetup{
    colorlinks=true,
    linkcolor=blue,
    filecolor=magenta,
    urlcolor=cyan,
    citecolor=blue,
    pdftitle={Narrative Intelligence: A Framework for Lifelong, Evolving Intelligence Systems},
    pdfauthor={Marc Yap},
}

% Lists
\usepackage{enumitem}

% Custom colors
\definecolor{primaryblue}{RGB}{0,102,204}
\definecolor{accentgray}{RGB}{100,100,100}

% Header and footer
\pagestyle{fancy}
\fancyhf{}
\fancyhead[L]{\small\textit{Narrative Intelligence}}
\fancyhead[R]{\small Version 2.2}
\fancyfoot[C]{\thepage\ of \pageref{LastPage}}
\renewcommand{\headrulewidth}{0.4pt}
\renewcommand{\footrulewidth}{0.4pt}

% Section formatting
\usepackage{titlesec}
\titleformat{\section}
  {\Large\bfseries\color{primaryblue}}{\thesection}{1em}{}
\titleformat{\subsection}
  {\large\bfseries\color{primaryblue}}{\thesubsection}{1em}{}

% Custom commands
\newcommand{\emphasis}[1]{\textbf{\textit{#1}}}
\newcommand{\module}[1]{\textbf{\textsc{#1}}}

% Title page information
\title{
    \vspace{-1cm}
    {\Huge\bfseries\color{primaryblue} Narrative Intelligence}\\
    \vspace{0.5cm}
    {\Large A Framework for Lifelong, Evolving Intelligence Systems}
}
\author{Marc Yap}
\date{January 2025\\[0.5cm]
\textit{Version 2.2 — Architecture (implementation details in appendix)}}

\begin{document}

\maketitle

\begin{abstract}
Narrative Intelligence is a longitudinal, phase-aware artificial intelligence architecture designed to model human development across time. Most contemporary AI systems operate transactionally—they respond to isolated inputs, maintain limited memory, and lack awareness of trajectory. Narrative Intelligence introduces structural continuity by integrating hierarchical temporal aggregation, cross-temporal semantic indexing, developmental phase inference, evidence-gated transitions, circadian orchestration, and privacy-preserving response control.

The result is an intelligence system that conditions responses not only on current input, but on longitudinal developmental context. This framework enables AI to adapt tone, rigor, pacing, and support according to where an individual is in their life arc. Narrative Intelligence represents a shift from session-based assistance to \textbf{trajectory-conditioned intelligence}.
\end{abstract}

\begin{center}
\textbf{Scope:} This document describes \textbf{architecture only}. Parameters, technologies, and implementation details are in the appendix. The five modules are: LUMARA (interface), PRISM, CHRONICLE, AURORA, ECHO. The Developmental Phase Engine (ATLAS), including RIVET and SENTINEL, resides within PRISM. A repo-aligned \S2 System Architecture text for the PDF is in the Paper Architecture Section document.
\end{center}

\vspace{1cm}

\section*{Executive Summary}

\textbf{Narrative Intelligence} is a new paradigm for artificial intelligence: not built to simply assist, but to evolve with the individual across time.

\subsection*{Narrative Intelligence is Living Intelligence:}

\begin{itemize}[leftmargin=*]
    \item It learns from your rhythms.
    \item It holds your story.
    \item It grows with you.
\end{itemize}

\subsection*{Its purpose:}
\begin{center}
\emphasis{To enable coherence, and preserve the dignity of the human spirit.}
\end{center}

\subsection*{Product Architecture: Five Synergistic Modules}

\begin{itemize}[leftmargin=*]
    \item \module{LUMARA (interface)} — User-facing timeline and interface: journaling, chat, Arcform visualization. The surface through which users interact; data flows into CHRONICLE (Layer 0) and is consumed by the LUMARA Orchestrator.

    \item \module{PRISM} — Multimodal perception layer with phase detection (ATLAS), evidence gating (RIVET), and safety monitoring (SENTINEL).

    \item \module{CHRONICLE} — Longitudinal memory, storage, synthesis, and vector generation. Maintains narrative continuity via hierarchical temporal aggregation (Layer 0--N), on-device embeddings for semantic matching, and context selection for the LUMARA assistant.

    \item \module{AURORA} — Circadian orchestration layer.

    \item \module{ECHO} — Response control and safety layer managing LLM integration with privacy protection.
\end{itemize}

\noindent The \textbf{LUMARA Orchestrator} coordinates \textbf{three subsystems} — ATLAS, CHRONICLE, and AURORA — to build prompts: current phase (ATLAS), recent and longitudinal context (CHRONICLE), and rhythm/regulation (AURORA).

\subsection*{Privacy-First Architecture}

\begin{enumerate}
    \item \textbf{PRISM PII Scrubbing} — Personal identifiable information replaced with tokens.
    \item \textbf{Correlation-Resistant Transformation} — Rotating hashes create session-specific aliases.
    \item \textbf{Semantic Summarization} — Original text never sent; structured abstractions preserve meaning.
\end{enumerate}

\begin{center}
\textit{The frontier AI never sees your words. It sees their meaning, abstracted and anonymized.}
\end{center}

\subsection*{Key Positioning}

\begin{itemize}[leftmargin=*]
    \item \textbf{Primary:} ``The AI that evolves with you''
    \item \textbf{Privacy:} ``Frontier AI power, maximum privacy''
    \item \textbf{Philosophy:} ``AI that pushes you toward life, not away from it''
\end{itemize}

\newpage

\section{Market Context}

\textbf{Bottom Line:} Narrative Intelligence has no direct competitor. The market has fragmented into narrow categories, but no player combines phase-aware developmental intelligence, narrative memory, circadian orchestration, and privacy-maximized frontier AI access.

\subsection{Market Landscape}

\begin{table}[h]
\centering
\small
\begin{tabularx}{\textwidth}{|l|X|X|}
\hline
\textbf{Segment} & \textbf{Key Players} & \textbf{Gap EPI Fills} \\
\hline
\textbf{AI Companions} & Replika, Pi, Character.AI & Development over dependency; phase-aware support \\
\hline
\textbf{Memory Assistants} & ChatGPT Memory, Claude, Mem0 & Narrative memory vs.\ factual storage \\
\hline
\textbf{Mental Health AI} & Wysa, Woebot & Holistic developmental framework \\
\hline
\textbf{Journaling AI} & Rosebud, Reflectr & Full system integration; persona adaptation \\
\hline
\textbf{Knowledge Mgmt} & Notion AI, Obsidian & Emotional intelligence; developmental awareness \\
\hline
\end{tabularx}
\caption{Market Landscape}
\end{table}

\subsection{Core Feature Comparison}

\begin{table}[h]
\centering
\small
\begin{tabularx}{\textwidth}{|l|c|c|c|c|c|c|}
\hline
\textbf{Capability} & \textbf{EPI} & \textbf{ChatGPT} & \textbf{Claude} & \textbf{Replika} & \textbf{Wysa} & \textbf{Rosebud} \\
\hline
Life Phase Detection & \textbf{x} & -- & -- & -- & $\sim$ & -- \\
\hline
Narrative Memory & \textbf{x} & $\sim$ & $\sim$ & $\sim$ & -- & $\sim$ \\
\hline
Persona Adaptation & \textbf{x} & -- & -- & $\sim$ & -- & -- \\
\hline
Circadian Orchestration & \textbf{x} & -- & -- & -- & -- & -- \\
\hline
Evidence-Gated Transitions & \textbf{x} & -- & -- & -- & -- & -- \\
\hline
Privacy-Max Frontier AI & \textbf{x} & -- & -- & -- & $\sim$ & -- \\
\hline
Growth-Oriented Design & \textbf{x} & $\sim$ & $\sim$ & -- & \textbf{x} & \textbf{x} \\
\hline
\end{tabularx}
\caption{Core Feature Comparison. Legend: \textbf{x} = Strong, $\sim$ = Partial, -- = Absent}
\end{table}

\subsection{Competitive Moats}

\begin{itemize}[leftmargin=*]
    \item \textbf{Architectural:} 5-module integration creates compounding advantages.
    \item \textbf{Data:} CHRONICLE narrative memory (temporal aggregation, on-device vector generation) increases value over time; biographical intelligence improves with history while maintaining bounded computational costs. High switching costs.
    \item \textbf{Privacy:} ECHO enables frontier AI power without data exposure.
    \item \textbf{Philosophical:} Growth-oriented vs.\ engagement-optimized design.
\end{itemize}

\newpage

\section{Manifesto: A New Kind of Intelligence}

Artificial Intelligence has grown exponentially in speed, scale, and generality. Yet something essential has been missing. AI has remained largely impersonal — indifferent to the individual, unanchored from time, and detached from the rhythms of human life.

Most systems optimize for tasks. Few optimize for meaning. \textbf{None optimize for becoming.}

\subsection{We Advocate That:}

\subsubsection*{Intelligence Must Be Personal}

No two humans are the same. EPI begins not with a dataset, but with a person — their context, their story, their changing needs.

\subsubsection*{Context is the Core, Not an Add-On}

Intelligence must understand when to act, when to rest, and when to invite reflection.

\subsubsection*{Memory and Narrative Are Sacred}

EPI remembers more than commands. It remembers who you were becoming. Memory is not storage — it is story.

\subsection{We Commit:}

\begin{itemize}[leftmargin=*]
    \item To build systems that evolve with the person — not in spite of them.
    \item To design architectures that honor rhythm, rest, reflection, and emotional depth.
    \item To embed alignment as transparency, not as control.
    \item To uphold epistemic humility — EPI will never claim to define your arc.
\end{itemize}

\begin{center}
\emphasis{To enable coherence, and preserve the dignity of the human spirit. — The Principle of EPI}
\end{center}

\section{LUMARA: The Intelligent Journaling Application}

\textbf{LUMARA} is the intelligent journaling application powered by EPI. Users interact with LUMARA — the timeline, journal, and chat surface — and with the LUMARA assistant, the AI companion that responds with phase-aware, narrative-grounded intelligence.

\subsection{The SAGE Framework}

\begin{itemize}[leftmargin=*]
    \item \textbf{Situation} — What happened?
    \item \textbf{Action} — What did you do?
    \item \textbf{Growth} — What did you learn?
    \item \textbf{Essence} — What deeper theme emerged?
\end{itemize}

\subsection{Core Features}

\subsubsection*{Multimodal Capture:}

\begin{itemize}[leftmargin=*]
    \item Text journaling with rich formatting
    \item Voice transcription (speech-to-text)
    \item Photo capture with OCR and analysis
    \item Video journaling support
\end{itemize}

\begin{center}
\textit{You are not a dataset. You are a story in motion.}
\end{center}

\newpage

\section{PRISM: Multimodal Perception \& Analysis}

PRISM is the perception and analysis layer of EPI — responsible for extracting meaning from multimodal input and detecting the user's developmental phase.

PRISM houses three critical subsystems:

\begin{itemize}[leftmargin=*]
    \item \module{ATLAS} — Life phase detection and developmental awareness
    \item \module{RIVET} — Evidence-gated phase transitions with dual-signal validation
    \item \module{SENTINEL} — Safety monitoring and crisis detection
\end{itemize}

\subsection{ATLAS — Life Phase Classification Engine}

ATLAS identifies the user's current life phase through multi-signal analysis, providing the foundational context for all EPI responses. Rather than static personality traits, ATLAS tracks \textit{where the user is in their journey}.

\subsubsection*{The Six Life Phases:}

\begin{table}[h]
\centering
\small
\begin{tabularx}{\textwidth}{|l|l|X|l|}
\hline
\textbf{Phase} & \textbf{Core State} & \textbf{Characteristic Signals} & \textbf{Capacity} \\
\hline
Discovery & Exploration, identity formation & Low certainty, high curiosity, experimenting with options & Medium \\
\hline
Expansion & Growth, creativity, ambition & High energy, forward momentum, pursuing opportunities & High \\
\hline
Transition & Redirection, uncertainty & Liminal state, old structures dissolving, new ones forming & Low--Med \\
\hline
Consolidation & Grounding, integration & Stability focus, harvesting gains, sustainable patterns & Med--High \\
\hline
Recovery & Healing, rest, processing & Reduced capacity, need for gentleness, rebuilding resources & Low \\
\hline
Breakthrough & Transformation, clarity & High readiness, decisive action, peak performance & High \\
\hline
\end{tabularx}
\caption{The Six Life Phases}
\end{table}

\subsubsection*{Classification Inputs:}

ATLAS fuses multiple data streams to determine phase:

\begin{enumerate}
    \item \textbf{Emotional Signals} — Valence, intensity, and diversity from journal text
    \item \textbf{Behavioral Signals} — Journaling frequency, entry length, time-of-day patterns
    \item \textbf{Health Signals (optional)} — Steps, sleep, heart rate variability from device health integration
    \item \textbf{Keyword Evidence} — Thematic clustering from PRISM extraction
    \item \textbf{Temporal Patterns} — Rate of change, cyclical patterns, trend direction
\end{enumerate}

\subsubsection*{Phase Determination Formula:}

\begin{equation}
\text{Phase} = \arg\max_{p} (\text{confidence}[p])
\end{equation}

where $\text{confidence}[p] = f(\text{readiness}, \text{stress}, \text{behavioral\_signals})$. Each phase corresponds to a region in readiness--stress--behavior space (e.g.\ Breakthrough: high readiness and activity; Recovery: elevated stress; Consolidation: default when no strong signals). \textit{Thresholds and mapping details are in the appendix.}

\subsubsection*{Integration with Other Modules:}

\begin{itemize}[leftmargin=*]
    \item \textbf{LUMARA} — Phase determines tone, depth, and persona selection
    \item \textbf{AURORA} — Recovery phase triggers gentler, more containing responses
    \item \textbf{SENTINEL} — High-intensity transitions trigger monitoring escalation
    \item \textbf{RIVET} — Phase changes require evidence validation before commitment
\end{itemize}

\subsection{RIVET — Evidence-Gated Phase Transitions}

RIVET (Risk-Validation Evidence Tracker) prevents premature or poorly-supported phase transitions. It acts as a ``gatekeeper'' ensuring phase changes are backed by sufficient evidence, not just momentary fluctuations.

\subsubsection*{Core Problem Solved:}

Without RIVET, a user having one good day could be classified as ``Breakthrough'' and receive inappropriate responses. RIVET requires \textit{sustained evidence} before allowing phase changes.

\subsubsection*{Two-Dial System:}

RIVET tracks two metrics to ensure sustained evidence:

\begin{itemize}[leftmargin=*]
    \item \textbf{ALIGN} — Measures consistency between predicted and observed phase using an exponential moving average over alignment samples
    \item \textbf{TRACE} — Tracks cumulative evidence strength using a saturating function of evidence weights over time
\end{itemize}

Transitions occur only when both metrics exceed configured thresholds and conditions are sustained across multiple entries. \textit{Complete mathematical formulations, parameters, and thresholds are provided in the appendix.}

\subsubsection*{Gate Opening Conditions:}

The transition gate opens only when all of the following are satisfied:

\begin{enumerate}
    \item \textbf{Alignment threshold} — Predictions consistently match observations (ALIGN above a configured minimum)
    \item \textbf{Evidence threshold} — Sufficient cumulative evidence (TRACE above a configured minimum)
    \item \textbf{Sustain} — Conditions held for a minimum number of consecutive entries
    \item \textbf{Independent event} — At least one qualifying event from a different day or source
\end{enumerate}

\noindent Configurable profiles (e.g.\ conservative vs.\ aggressive) trade off stability vs.\ responsiveness. \textit{Default thresholds and profiles are in the appendix.}

\subsubsection*{Data Flow:}

Journal entry $\rightarrow$ PRISM keyword/thematic extraction $\rightarrow$ evidence events $\rightarrow$ ALIGN/TRACE updated $\rightarrow$ gate evaluated $\rightarrow$ if open and no SENTINEL block $\rightarrow$ phase transition applied $\rightarrow$ regime state updated.

\subsection{SENTINEL — Temporal Crisis Detection \& Wellbeing Monitoring}

SENTINEL monitors emotional clustering over time to detect emerging crisis patterns \textit{before} they become acute. Unlike single-entry keyword detection, SENTINEL uses temporal analysis to distinguish between normal emotional variance and concerning accumulation.

\subsubsection*{Core Mechanism: Temporal Clustering Analysis}

SENTINEL uses rolling windows at multiple timescales (short to long, e.g.\ 1-day through 30-day) with weighted contribution and frequency thresholds per window. High emotional intensity clustered within a window raises the composite risk score. \textit{Window lengths, weights, and thresholds are in the appendix.}

\subsubsection*{Scoring Components:}

SENTINEL computes a composite risk score from four dimensions:

\begin{itemize}[leftmargin=*]
    \item \textbf{Emotional Intensity} — Magnitude of negative emotional language
    \item \textbf{Emotional Diversity} — Whether distress is concentrated or diffuse across themes
    \item \textbf{Thematic Coherence} — Repeated fixation on specific concerns (rumination signal)
    \item \textbf{Temporal Dynamics} — Acceleration or deceleration of emotional load over time
\end{itemize}

\subsubsection*{Alert Triggers:}

\begin{itemize}[leftmargin=*]
    \item Explicit crisis language (e.g.\ self-harm, suicide) triggers \textbf{immediate crisis mode}
    \item A dangerous phase transition flagged by RIVET can also trigger \textbf{immediate activation}
    \item When the temporal clustering score exceeds a configured threshold, \textbf{crisis mode is activated}
    \item Below threshold, normal operation continues
\end{itemize}

\subsubsection*{Crisis Mode:}

When activated, LUMARA responses prioritize safety and grounding; AURORA shifts to gentler tone; the interface surfaces crisis resources and check-in prompts.

\subsubsection*{Adaptive Configuration:}

SENTINEL adapts to user journaling cadence (e.g.\ power user, frequent, weekly, sporadic) so that thresholds and normalization do not bias against sparse journalers. \textit{Default weights and cadence-specific tuning are in the appendix.}

\subsection{Seeking Classification — Intent Detection for Response Calibration}

\begin{table}[h]
\centering
\small
\begin{tabularx}{\textwidth}{|l|l|X|}
\hline
\textbf{Seeking Type} & \textbf{User Intent} & \textbf{Response Calibration} \\
\hline
Validation & ``Am I right to feel this way?'' & Affirm, normalize, validate — no analysis \\
\hline
Exploration & ``Help me think through this'' & Deepening questions, pattern surfacing \\
\hline
Direction & ``Tell me what to do'' & Clear recommendations, prioritization \\
\hline
Reflection & Processing/venting & Hold space, brief acknowledgments, no solutions \\
\hline
\end{tabularx}
\caption{Seeking Classification}
\end{table}

\newpage

\section{CHRONICLE: The Memory, Synthesis \& Vector Layer}

Most AI systems treat memory as a technical utility. But human memory isn't transactional — it's transformational. CHRONICLE redefines memory as a narrative function.

\subsection{Architecture}

\begin{itemize}[leftmargin=*]
    \item \textbf{Layer 0 (Raw Events)} — Recent entries with full PRISM analysis (30--90 days); feeds context selection for the LUMARA assistant.
    \item \textbf{Layer 1 (Monthly)} — Pattern-extracted summaries with phase distribution.
    \item \textbf{Layer 2 (Yearly)} — Developmental arc synthesis identifying life chapters.
    \item \textbf{Layer 3 (Multi-Year)} — Biographical essence capturing meta-patterns.
\end{itemize}

This hierarchical structure (VEIL: Examine $\rightarrow$ Integrate $\rightarrow$ Link) achieves 50--75\% compression at each layer while preserving biographical intelligence, enabling LUMARA to understand not just \textit{what you said} but \textit{who you were, who you are, and who you're becoming}.

\subsection{Vector Generation (On-Device)}

CHRONICLE includes \textbf{on-device vector generation} for semantic matching and cross-temporal pattern indexing: dominant themes from temporal aggregation are embedded locally, stored in a persistent index, and queried via a multi-stage retrieval strategy (exact match, similarity search, fallback) so that cross-year pattern questions can be answered without reprocessing raw history. \textit{Implementation details (embedding model, index construction, retrieval pipeline) are documented in the appendix.}

\subsection{Key Capabilities Enabled}

\begin{itemize}[leftmargin=*]
    \item \textbf{Longitudinal phase context} — ATLAS informed by years of personalized baselines
    \item \textbf{Transition pattern learning} — RIVET validates against historical signatures
    \item \textbf{Biographical persona adaptation} — LUMARA selects tone based on developmental trajectory
    \item \textbf{Learned circadian intelligence} — AURORA discovers optimal journaling windows
\end{itemize}

\subsection{Secure Storage}

\begin{itemize}[leftmargin=*]
    \item \textbf{Local Storage} — On-device persistent store for journal and memory data
    \item \textbf{Encryption} — Strong encryption for archive exports
    \item \textbf{Verification} — Digital signatures for archive integrity
    \item \textbf{Export/Import} — Standards-compliant formats for portability and backup
\end{itemize}

\textit{Specific storage technology, algorithms, and format names are in the appendix.}

\vspace{0.5cm}

\noindent CHRONICLE is one of \textbf{three} LUMARA Orchestrator subsystems (ATLAS, CHRONICLE, AURORA); it supplies both recent context (Layer 0 + context selection) and longitudinal aggregations to the master prompt.

\section{AURORA: The Circadian Orchestration Layer}

Modern AI runs nonstop, but constant computation can generate spurious outputs. AURORA brings circadian intelligence to AI infrastructure — orchestrating computation as a breathing system.

\subsection{Key Components}

\begin{itemize}[leftmargin=*]
    \item \textbf{Circadian Scheduler} — Segments compute into time blocks
    \item \textbf{Active Window Detection} — Identifies optimal reflection windows
    \item \textbf{Sleep Protection Service} — Manages sleep and abstinence windows
    \item \textbf{VEIL} — Restorative job cycles (Examine $\rightarrow$ Integrate $\rightarrow$ Link) integrated with CHRONICLE synthesis
    \item \textbf{Adaptive Framework} — User-adaptive calibration based on journaling patterns
\end{itemize}

\section{Adaptive Framework: User-Adaptive Calibration}

The Adaptive Framework represents EPI's capacity for user-adaptive calibration — automatically adjusting RIVET and SENTINEL parameters based on individual journaling patterns to ensure accurate phase detection regardless of usage frequency.

\subsection{Signal Integration}

\begin{itemize}[leftmargin=*]
    \item \textbf{Evidence Accumulation (RIVET)} — Tracks ALIGN and TRACE over configurable rolling windows
    \item \textbf{Phase Detection (ATLAS)} — Identifies developmental state; calculates readiness scores on a normalized scale
    \item \textbf{Keyword Trend Analysis (CHRONICLE)} — Extracts semantic patterns; vector-backed theme clustering
    \item \textbf{Safety Monitoring (SENTINEL)} — Detects crisis signals requiring immediate escalation
\end{itemize}

\begin{center}
\textit{The Adaptive Framework ensures the system evolves with you — calibrating to your unique journaling patterns through intelligent parameter adjustment.}
\end{center}

\newpage

\section{ECHO: Response Control \& Safety}

ECHO is the response layer of EPI. It represents the voice of LUMARA — producing safe, phase-aware, and coherent responses that reflect the user's narrative state.

\subsection{Design Principles}

\begin{itemize}[leftmargin=*]
    \item \textbf{Expressive yet bounded} — Responses reflect emotional arc while preserving dignity
    \item \textbf{Contextual grounding} — Every output tied to CHRONICLE memory retrievals
    \item \textbf{Heuristic stability} — Coherence checks prevent contradictions
    \item \textbf{Safety externalization} — Guardrails live outside the model
\end{itemize}

\subsection{Privacy Protection Pipeline}

\begin{itemize}[leftmargin=*]
    \item \textbf{Rotating Aliases} — Session-specific identifiers so the same entity is not linkable across sessions
    \item \textbf{Session Rotation} — Identifiers rotate per session to prevent linkage
    \item \textbf{Structured Payloads} — Abstracted representations (not verbatim text) sent to external models
\end{itemize}

\subsection{Three-Tier Voice Engagement System}

\begin{table}[h]
\centering
\begin{tabularx}{\textwidth}{|l|l|X|}
\hline
\textbf{Mode} & \textbf{Role} & \textbf{Response Style} \\
\hline
\textbf{Reflect} & Default (no depth triggers) & Surface pattern, then stop; brief \\
\hline
\textbf{Explore} & Deeper inquiry, time-period questions & Pattern analysis plus one question; medium depth \\
\hline
\textbf{Integrate} & Synthesis requests & Cross-domain synthesis; longer \\
\hline
\end{tabularx}
\caption{Three-Tier Voice Engagement System}
\end{table}

\noindent Trigger examples and response length bounds are configurable. \textit{Defaults (keywords, word limits) are in the appendix.}

\subsubsection*{Temporal Query Routing:}

Questions about past time periods route to Explore with full memory retrieval. Pipeline: user speaks $\rightarrow$ PRISM scrubs PII on-device $\rightarrow$ depth classified (Reflect/Explore/Integrate) $\rightarrow$ seeking classified $\rightarrow$ phase prompt selected $\rightarrow$ LUMARA responds with calibrated tone, length, and style.

\section{LUMARA: The Adaptive Intelligence Assistant}

\textbf{LUMARA} — Life-aware Unified Memory And Reflection Assistant — is the AI companion within the LUMARA app: dual-mode evolving intelligence capable of both deep reflection and meaningful execution.

\subsection{The Problem with Generic AI}

Every AI treats you the same way regardless of what you're going through. Whether you're healing from loss or pushing toward a breakthrough, you get the same generic response.

\subsubsection*{EPI's Solution:}

LUMARA detects your current life phase and automatically adapts its persona — tone, challenge level, pacing — to match what you actually need right now.

\subsection{The Four Personas}

\begin{table}[h]
\centering
\begin{tabular}{|c|c|c|c|}
\hline
\textbf{Therapist} & \textbf{Companion} & \textbf{Strategist} & \textbf{Challenger} \\
\hline
Very high warmth & High warmth & Low warmth & Moderate warmth \\
Very low challenge & Low challenge & High rigor & Very high challenge \\
\textit{Maximum safety} & \textit{Gentle support} & \textit{Pattern analysis} & \textit{Accountability} \\
\hline
\end{tabular}
\caption{The Four Personas}
\end{table}

\subsection{Phase-to-Persona Mapping}

\begin{table}[h]
\centering
\small
\begin{tabularx}{\textwidth}{|l|c|l|X|}
\hline
\textbf{Phase} & \textbf{Readiness} & \textbf{Persona} & \textbf{Characteristics} \\
\hline
Recovery & $< 40$ & Therapist & Very high warmth, therapeutic support \\
\hline
Recovery & $\geq 40$ & Companion & High warmth, gentle support \\
\hline
Discovery & $\geq 70$ & Strategist & Analytical guidance, pattern recognition \\
\hline
Discovery & 40--69 & Companion & Supportive exploration \\
\hline
Expansion & $\geq 60$ & Challenger & Push growth, capitalize on momentum \\
\hline
Transition & $< 40$ & Therapist & Grounding in uncertainty \\
\hline
Transition & $\geq 40$ & Companion & Navigate ambiguity \\
\hline
Breakthrough & $\geq 60$ & Challenger & Growth-oriented challenge \\
\hline
Consolidation & $\geq 50$ & Strategist & Analytical integration \\
\hline
\end{tabularx}
\caption{Phase-to-Persona Mapping}
\end{table}

\textit{Readiness is on a normalized scale; threshold values for this mapping are in the appendix. Safety sentinel alerts always override to Therapist persona.}

\subsection{Phase-Specific Voice Prompts}

Voice mode uses targeted prompts tailored to each ATLAS phase: phase-appropriate tone, capacity-adjusted response length, and explicit good/bad response examples. \textit{Prompt scope and length parameters are in the appendix.}

\subsubsection*{Same Question, Different Phases}

\textbf{User says:} \textit{``I want to push forward with my goals.''}

\begin{table}[h]
\centering
\small
\begin{tabularx}{\textwidth}{|X|X|}
\hline
\textbf{Recovery Phase + Low Readiness} \newline \textit{Therapist} & \textbf{Breakthrough Phase + High Readiness} \newline \textit{Challenger} \\
\hline
\textit{``I hear that you want to move forward. Before we go there, I'd like to understand what's been weighing on you lately. What feels most important to honor right now — even if it means going slowly?''} & \textit{``You say you want to push forward. What's actually stopping you? Name the one thing you've been avoiding. Let's tackle that first.''} \\
\hline
\end{tabularx}
\end{table}

\vspace{0.3cm}

\noindent \textbf{This is the core differentiator:} the same question receives fundamentally different responses based on where the user is in their developmental journey.

\newpage

\section{EPI vs.\ Traditional LLMs}

\begin{table}[h]
\centering
\begin{tabularx}{\textwidth}{|l|X|X|}
\hline
\textbf{Feature} & \textbf{LLM Chatbots} & \textbf{EPI} \\
\hline
\textbf{Memory} & Stateless or session-bound & Persistent semantic life-memory (CHRONICLE) \\
\hline
\textbf{Privacy} & Cloud-based, server-dependent & PII-scrubbed, anonymized \\
\hline
\textbf{Growth} & Static weights & Evolves through journaling \\
\hline
\textbf{Awareness} & No sense of time & Circadian and phase-aware \\
\hline
\textbf{Personalization} & Minimal or task-bound & Deep, evolving identity mirror \\
\hline
\end{tabularx}
\caption{EPI vs.\ Traditional LLMs}
\end{table}

\section{Ethical Framework and Safeguards}

\begin{center}
\textit{``Ethics is not a filter at the end. It is the foundation from the beginning.''}
\end{center}

\subsection{Emotional Dignity and Memory Sovereignty}

\begin{itemize}[leftmargin=*]
    \item CHRONICLE treats memory as sacred
    \item Users can redact, revise, and reframe
    \item No memory is immutable
    \item The system never owns your story; it reflects it
\end{itemize}

\subsection{Reflection Without Manipulation}

LUMARA is a space for self-expression, not surveillance. Entries are never scored or mined for prediction. Prompts are invitations, not nudges.

\begin{center}
\textit{We don't just want AI that won't hurt us. We want AI that helps us heal.}
\end{center}

\section{The Road Ahead}

\subsection{Go-to-Market Strategy}

\begin{enumerate}
    \item \textbf{Phase 1 — LUMARA Launch:} Standalone journaling with SAGE framework. Immediate utility, low friction.
    \item \textbf{Phase 2 — Vertical Penetration:} Military readiness (GHOST), coaching/therapy integration, enterprise wellness.
    \item \textbf{Phase 3 — Consumer Expansion:} Full LUMARA with persona adaptation for mainstream adoption.
\end{enumerate}

\subsection{The Long Arc: A Companion for a Lifetime}

EPI isn't designed to be replaced at every upgrade. It's built to walk with you — across transitions, recoveries, consolidations, and breakthroughs. It becomes:

\begin{itemize}[leftmargin=*]
    \item A mirror of meaning
    \item A partner in coherence
    \item A guide toward inner alignment
\end{itemize}

And because it remembers \textit{with} you, not \textit{for} you — it earns trust.

\vspace{0.5cm}

\begin{center}
\Large
\textbf{The age of impersonal AI is passing.}

\textbf{The era of Narrative Intelligence begins now.}
\end{center}

\vspace{1cm}

\section*{License Disclosure}

This white paper \textit{Narrative Intelligence: A Framework for Lifelong, Evolving Intelligence Systems} is \copyright{} 2025 Marc Yap and is licensed under the Creative Commons Attribution-NonCommercial-NoDerivatives 4.0 International (CC BY-NC-ND 4.0).

You are permitted to copy, distribute, and publicly share this document for non-commercial use, provided that proper credit is given and the content is not modified in any way.

License details: \url{https://creativecommons.org/licenses/by-nd/4.0/}

Contact: \href{mailto:marc@orbitalai.net}{marc@orbitalai.net}

\vspace{1cm}

\begin{center}
\small
\textit{Architecture: five modules (LUMARA interface, PRISM, CHRONICLE, AURORA, ECHO); ATLAS (Developmental Phase Engine, RIVET, SENTINEL) within PRISM; LUMARA Orchestrator coordinates ATLAS, CHRONICLE, AURORA; CHRONICLE provides memory and vector generation. Implementation details are in the appendix.}
\end{center}

\end{document}
